% compile with xelatex
\documentclass{article} 
\usepackage{polyglossia} 
\usepackage{amsmath}
\usepackage{fontspec} 
\usepackage{lipsum} 
\usepackage[margin=1in]{geometry}
\usepackage{graphicx} 
\usepackage{caption} 
\usepackage{subcaption}
\usepackage{hyperref} 
\usepackage{booktabs}
%\usepackage{listing}


%%% Metadata and link colors %%% 
\hypersetup{% 
    colorlinks=true, linkcolor=magenta, filecolor=yellow,      
    urlcolor=blue, 
    pdfinfo = {%
        Title = Εργασία Βαθιάς Μάθησης
        Author = {Χρήστος Μάριος Περδίκης},
        Producer = XeLaTeX,
    } 
}

%%% Title, language %%% 
\title{Ανάλυση Ιατρικών Εικόνων MedMNIST με CNN, Transfer Learning \& Vision Transformers (PyTorch)}
\date{Εργασία Βαθιάς Μάθησης Χειμερινό Εξάμηνο 2025-2026}
\author{Χρήστος Μάριος Περδίκης 10075 cperdikis@ece.auth.gr}

%%% font stuff, languages, document appearance %%% 
\setmainlanguage{greek}     % this is super useful, automatically changes "Listing 1" and "Table 1" to "Σχήμα 1" and "Πίνακας 1". Super useful stuff.
\setotherlanguage{english}

\setmainfont{CMU Serif}
%\setmainfont{FreeSans}
%\newfontfamily\greekfonttt[Script=Greek]{FreeMono Bold}[SizeFeatures={Size=9}] 

\setlength{\parindent}{0pt}
\setlength{\parskip}{1em}

%%% custom commands and variable %%% 
\NewDocumentCommand{\datasetPicSize}{}{28}
\NewDocumentCommand{\datasetPlithosEikonwn}{}{112,120}
\NewDocumentCommand{\datasetTrainingPlithos}{}{78,468}
\NewDocumentCommand{\datasetValidationPlithos}{}{11,219}
\NewDocumentCommand{\datasetTestPlithos}{}{22,433}
\NewDocumentCommand{\trainingOverTotalPerc}{}{69.98}
\NewDocumentCommand{\validationOverTotalPerc}{}{10}
\NewDocumentCommand{\testOverTotalPerc}{}{20.01}

\begin{document} 
\maketitle 

\section{Περιγραφή dataset}
Το dataset που χρησιμοποιήθηκε είναι το ChestMNIST της MedMNIST με μέγεθος εικόνων \datasetPicSize{}x\datasetPicSize{}.
οι εικόνες είναι ασπρόμαυρες. Το dataset αποτελείται από συνολικά από \datasetPlithosEikonwn{} εικόνες. Χωρίζεται σε training,
validation και test splits. Στο training split ανήκουν \datasetTrainingPlithos{} εικόνες, δηλαδή το
$\frac{\datasetTrainingPlithos{}}{\datasetPlithosEikonwn{}} = \trainingOverTotalPerc{}\%$ των συνολικών δεδομένων. 
Στο validation split ανήκουν \datasetValidationPlithos{} εικόνες δηλαδή το 
$\frac{\datasetValidationPlithos{}}{\datasetPlithosEikonwn{}} = 
\validationOverTotalPerc{}\%$ των συνολικών δεδομένων. Στο test split ανήκουν \datasetTestPlithos{} εικόνες δηλαδή το 
$\frac{\datasetTestPlithos{}}{\datasetPlithosEikonwn{}} = \testOverTotalPerc{}\%$ των συνολικών δεδομένων. Η ανισορροπία
στα τρία splits παρατηρείται γιατί ένα το training ενός Deep Learning Network χρειάζεται πολλά παραπάνω δεδομένα για την
εκπαίδευση από τα validation και test.

Οι κλάσεις στις οποίες είναι χωρισμένο το dataset περιγράφουν κοινές παθήσεις στην περιοχή του θώρακα, είτε στους πνεύμονες
ή στην καρδιά.  Ακολουθεί λίστα με τις κλάσεις και σύντομη περιγραφή της κάθε πάθησης, καθώς και πώς αυτή εμφανίζεται σε μια
ακτινογραφία θώρακος:
\begin{itemize}
    \item '0': 'atelectasis'. Η Ατελεκτασία	είναι μερική ή πλήρης σύμπτωση ενός πνεύμονα ή ενός λοβού του πνεύμονα, 
        συχνά λόγω απόφραξης (όπως βλέννα) ή πίεσης έξω από τον πνεύμονα Σε ακτινογραφία εμφανίζεται ως	αυξημένη πυκνότητα 
        (λευκότητα) στην επηρεαζόμενη περιοχή.
    \item '1': 'cardiomegaly'.  Η Καρδιομεγαλία χαρακτιρίζεται από τη διεύρυνση της καρδιάς, μπορεί να είναι σημάδι καρδιακής 
        ανεπάρκειας, προβλημάτων βαλβίδων ή υψηλής αρτηριακής πίεσης. Σε μια ακτινογραφία φαίνεται η σιλουέτα της καρδιάς να είναι 
        διευρυμένη.
    \item '2': 'effusion'. Η Πλευριτική Συλλογή αναφέρεται σε μια μη φυσιολογική συσσώρευση υγρού στον πλευ\-ριτικό χώρο 
        (ο χώρος μεταξύ του πνεύμονα και του θωρακικού τοιχώματος). Εμφανίζεται ως αμβλύτητα στις οξείες γωνίες μεταξύ των 
        πλευρών και του διαφράγματος, ένα σημείο μηνίσκου (καμπύλο άνω όριο του υγρού) και αυξημένη λευκότητα.
    \item '3': 'infiltration'. Η Διήθηση είναι ένας μη ειδικός όρος που χρησιμοποιείται συχνά για να περιγράψει οποιαδήποτε ουσία 
        που είναι πυκνότερη από τον αέρα και γεμίζει ένα τμήμα του πνεύμονα. Εμφανίζεται ως μια ασαφής ή κηλιδωτή περιοχή αυξημένης 
        λευκότητας.
    \item '4': 'mass'. Αναφέρεται σε μια μεγάλη, εστιακή πνευμονική βλάβη ή μη φυσιολογική ανάπτυξη. Τα όριά της (ομαλά ή λοβωτά) 
        και η θέση της (κεντρική ή περιφερειακή) είναι βασικά χαρακτηριστικά.
    \item '5': 'nodule'. Το Οζίδιο είναι μια μικρή, εστιακή πνευμονική βλάβη ή μη φυσιολογική ανάπτυξη, που ορίζεται τυπικά ως $<3 cm$ 
        σε διάμετρο. Εμφανίζεται ως μια μικρή, στρογγυλή ή οβάλ ακτινοσκιάνωση (λευκότητα). Χαρακτηριστικά όπως η ασβεστοποίηση ή 
        ο ρυθμός ανάπτυξης βοηθούν στη διάκριση καλοήθων από κακοήθη ευρήματα.
    \item '6': 'pneumonia'. Η Πνευμονία είναι φλεγμονή του πνεύμονα που προκαλείται κυρίως από λοίμωξη, όπου οι κυψελίδες 
        (αεροθάλαμοι) γεμίζουν με πύον και υγρό. Μπορεί να εμφανιστεί ως συμπαγοποίηση (στερε\-οποίηση του πνευμονικού ιστού) σε λοβώδη 
        κατανομή, ή ως πιο κηλιδωτές διηθήσεις.
    \item '7': 'pneumothorax'. Ο Πνευμοθώρακας χαρακτηρίζεται από Αέρα στον πλευριτικό χώρο, προκαλώντας μερική ή πλήρη 
        κατάρρευση του πνεύμονα. Εμφανίζεται ως μια ορατή, λεπτή λευκή γραμμή, με απουσία πνευμονικών αγγειακών σκιάσεων 
        (υπερδιαφάνεια/μαύρο).
    \item '8': 'consolidation'. Η Συμπαγοποίηση είναι μια διαδικασία όπου ο κυψελιδικός αέρας αντικαθίσταται από υγρό, φλεγμονώδες 
        εξίδρωμα ή άλλα προϊόντα (π.χ. σε πνευμονία). Εμφανίζεται ως μια περιοχή ομοιογενούς λευκότητας που δεν προκαλεί 
        απώλεια όγκου. Ο βρογχικός αέρας είναι ορατός εντός της αδιαφανούς περιοχής, κάτι που ονομάζεται αεροβρογχόγραμμα.
    \item '9': 'edema'. Το Πνευμονικό Οίδημα αναφέρεται στην υπερβολική συσσώρευση υγρού στους αεροθάλαμους  και το διάμεσο του πνεύμονα, 
        συχνά λόγω καρδιακής ανεπάρκειας. Εμφανίζεται ως λευκότητες που περιγράφονται ως μοτίβο φτερών νυχτερίδας ή πεταλούδας.
    \item '10': 'emphysema'. Το Εμφύσημα είναι ένας τύπος ΧΑΠ (Χρόνιας Αποφρακτικής Πνευμονοπάθειας) που χαρακτηρίζεται από την 
        καταστροφή των αεροθάλαμων, οδηγώντας σε διευρυμένους αεροχώρους και μειωμένη ανταλλαγή αερίων. 
        Εμφανίζεται ως μειωμένες αγγειακές σκιάσεις, και μερικές φορές παρουσία φυσαλίδων.
    \item '11': 'fibrosis'. Η Ίνωση περιγράφει την ουλοποίηση του πνευμονικού ιστού, συχνά ως αποτέλεσμα χρόνιας φλεγμονής, 
        καθιστώντας τον πνεύμονα δύσκαμπτο και λιγότερο ικανό να διασταλεί. Εμφανίζεται ως δικτυωτές ή γραμμικές λευκότητες, 
        μερικές φορές μαζί με μικρούς κυστικούς αερόχωρους.
    \item '12': 'pleural'. Η Πλευριτική Πάχυνση  χρησιμοποιείται συχνά ως γενικός όρος για μη ειδική πλευριτική νόσο, που είναι 
        ουλές της επένδυσης του πνεύμονα/θωρακικού τοιχώματος. Μπορεί να εμφανιστεί ως γραμμικές ακτινοσκιάνσεις κατά μήκος του 
        εσωτερικού θωρακικού τοιχώματος.
    \item '13': 'hernia'. Η Διαφραγματοκήλη συμβαίνει όταν ένα κοιλιακό όργανο (όπως το στομάχι ή τα έντερα) προβάλλει μέσω ενός 
        ανοίγματος στο διάφραγμα μέσα στη θωρακική κοιλότητα. Η εμφάνιση στην ακτινο\-γραφία εξαρτάται από το περιεχόμενο, αλλά 
        συχνά παρουσιάζεται ως μια μάζα ή αεροφόρος δομή πάνω από το διάφραγμα.
\end{itemize}
Στην εικόνα~\ref{pic_montage} υπάρχουν μερικά παραδείγματα εικόνων. 

\begin{figure}
    \centering
    \includegraphics[width=\textwidth]{pic_montage.png}
    \caption{Παραδείγματα εικόνων \datasetPicSize{}x\datasetPicSize{} ακτινογραφίας θώρακα από το dataset ChestMNIST}\label{pic_montage}
\end{figure}

Στον πίνακα~\ref{class_counts} υπάρχει ο αριθμός των εικόνων
που ανήκει σε κάθε κλάση. Αξίζει να σημειωθεί ότι το ChestMNIST είναι multi-label dataset, δηλαδή μια εικόνα μπορεί να ανήκει
σε παραπάνω από μια κλάσεις ταυτόχρονα και μερικές εικόνες μπορεί να μην ανήκουν και σε καμία κλάση. Είναι προφανές ότι
υπάρχει μεγάλη ανισορροπία στον αριθμό των εικόνων που ανήκουν σε κάθε κλάση. Στην κλάση `hernia' ανήκουν μόνο 227 εικόνες,
ενώ στις περισσότερες κλάσεις ανήκουν μερικές χιλιάδες λέξεις και σε μερικές κλάσεις ανήκουν δεκάδες χιλιάδες εικόνες.

\begin{table}
    \centering
    \begin{tabular}{lc}
        \toprule
        Class Name & Number of occurrences \\
        \midrule
        0 \hspace{1pt} 'atelectasis'         & 11535 \\
        1 \hspace{1pt} 'cardiomegaly'        & 2772 \\
        2 \hspace{1pt} 'effusion'        & 13307 \\
        3 \hspace{1pt} 'infiltration     & 19870 \\
        4 \hspace{1pt} 'mass'        & 5746 \\
        5 \hspace{1pt} 'nodule'      & 6323 \\
        6 \hspace{1pt} 'pneumonia'      & 1353 \\
        7 \hspace{1pt} 'pneumothorax'   & 5298 \\
        8 \hspace{1pt} 'consolidation'                & 4667 \\
        9 \hspace{1pt} 'edema'                & 2303 \\
        10 'emphysema'                & 2516 \\
        11 'fibrosis'                & 1686 \\
        12 'pleural'                & 3385 \\
        13 'hernia'                & 227 \\
        \bottomrule
    \end{tabular}\label{class_counts}
    \caption{Αριθμός εμφανίσεων της κάθε κλάσης σε όλο το dataset (training και test)}
\end{table}

\section{Μάλλον κώδικας}
\lipsum{2}

%%% ΚΕΦΑΛΑΙΟ 2 - CNN FROM SCRATCH %%%
%• Σχεδιάστε ένα CNN με τουλάχιστον 3 συνελικτικά blocks, κάθε block να περιλαμβάνει
%Conv2d → ReLU → MaxPooling2d. Προτείνεται να ξεκινήσετε με 32, 64 και 128 φίλτρα
%αντίστοιχα, kernel size 3×3. 
%• Προσθέστε ένα fully connected layer στο τέλος με αριθμό νευρώνων ίσο με τον αριθμό
%των κλάσεων.
%• Πειραματιστείτε με Batch Normalization μετά από κάθε Conv2d και/ή Layer
%Normalization στο τέλος κάθε block.
%• Δοκιμάστε Dropout στο fully connected layer με ποσοστά, 0.2, 0.5 και 0.7.
%• Χρησιμοποιήστε Weight Decay (L2 regularization) με τιμές 1e-4, 1e-3 και 1e-2.
%• Χρησιμοποιήστε Adam optimizer, με learning rate ενδεικτικά 1e-3, αλλά μπορείτε να
%πειραματιστείτε με τιμές μεταξύ 1e-4 και 5e-3.
%    o Χωρίς normalization ή dropout: 1e-3
%    o Με BatchNorm ή LayerNorm: 5e-4
%    o Με Dropout ή Weight Decay: 1e-3 ή 1e-4 ανάλογα με το συνδυασμό τεχνικών
%• Batch size: 64, Epochs: 30–50.
%• Παρακολουθήστε training και validation curves (loss & accuracy) για κάθε συνδυασμό
%τεχνικών.
%• Υπολογίστε test accuracy και δημιουργήστε confusion matrix.
%• Συγκρίνετε τα αποτελέσματα και σχολιάστε πώς οι διαφορετικές τεχνικές επηρεάζουν
%την απόδοση και τη γενίκευση.



%%% ΚΕΦΑΛΑΙΟ 1 - ΠΕΡΙΓΡΑΦΗ ΤΟΥ DATASET %%%
%• Περιγράψτε τον τύπο των εικόνων (π.χ. grayscale ή RGB) και την ανάλυση (π.χ. 28×28 ή 32×32). - done
%• Αναφέρετε τον αριθμό των κλάσεων και τη σημασία τους (π.χ. τύποι όγκων, κατηγορίες ιστών, κλπ.). - done
%• Καταγράψτε το μέγεθος του dataset (train/test/validation splits) και αν υπάρχει ανισορροπία στις κλάσεις. - done, να περιγράψω τον τρόπο με τον οποίο βρήκα των αριθμό των εικόνων σε κάθε κλάση;
%• Συμπεριλάβετε μερικά παραδείγματα εικόνων. - done, μήπως να βάλω παράδεγμα μια εικόνα που ανήκει σε κάθε κλάση;

\vspace{3em}
\centering
\emph{*** ΤΕΛΟΣ ΑΝΑΦΟΡΑΣ ***}
\end{document}
